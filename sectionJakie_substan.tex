\section{+Jakie substancje mogą pojawić na wylocie z układu a na wlocie do
SNG–CO2}

Na wlocie do układu CO2–SNG fizycznie może się pojawić wyłącznie mieszanina
H\textsubscript{2}/CO\textsubscript{2}/H\textsubscript{2}O. W~przypadku
awarii ogniw paliwowych (przeciek gazów z~elektrody tlenowej na~paliwową,
mogą się pojawić gazy znajdujące się w~spalinach, przy czym tlen
raczej ulegnie reakcji z~wodorem. Nie dopuszczamy takiej sytuacji,
można ona być łatwo kontrolowana poprzez utrzymanie lekkiego nadciśnienia
w~przepływie paliwowym (anoda), tak że przeciek nastąpi w~przeciwnym
kierunku (wodór do spalin).

Założenie projektowe: utrzymać nadciśnienie w~przepływie anodowym
w~stosunku do katodowego.